\documentclass{proposal}
\usepackage{zhlipsum}

\newcommand{\dif}{\mathrm{d}}

\hainnuset{%
	% -- 中文信息                        % 论文题目↓↓↓
	organization  = {海南师范大学},
	title         = {使用{\LaTeX{}}模板的开题报告撰写手册},
	studentid     = {2019xxxxx},        % 作者学号
	author        = {张三},              % 作者姓名
	degreelevel   = {本科},
	department	  = {信息科学技术学院},
	major		  = {计算机科学与技术},
	supervisor    = {李四},              % 指导教师姓名, 可以是两个
	acadetitle    = {副教授},            % 导师学术头衔
	submitdate    = {2022年9月9日}, 		% 填表日期
	purpose       = {毕业},   			% 文档目的:毕业/学位
	type		  = {设计},				% 文档类型:论文/设计, 由指导教师确定
}
\begin{document}
\linenumbers
%%%%%%%%%%%%%%%%%%%%%%%%%%%%%%%%%%%%%%%%%%%%%%%%%%%%%%%%%
%% 第一部分: 选题类别: 由下面命令的参数来自动生成,三选一.
%%%%%%%%%%%%%%%%%%%%%%%%%%%%%%%%%%%%%%%%%%%%%%%%%%%%%%%%%%%
\section{选题的类别}
\ResearchType{应用研究}			%选题类别:基础研究、应用研究、应用理论研究
\SectionEndLine				   %每一小节结束时,使用此命令生成表格结束的横线

%%%%%%%%%%%%%%%%%%%%%%%%%%%%%%%%%%%%%%%%%%%%%%%%%%%%%%%%%
%% 第二部分: 选题依据与研究意义
%%%%%%%%%%%%%%%%%%%%%%%%%%%%%%%%%%%%%%%%%%%%%%%%%%%%%%%%%%%

% 在这里撰写选题依据
\section{选题依据及研究意义}
\zhlipsum[2]
\subsection{论文概况}
\zhlipsum[1]

\subsection{自己填个小节的标题}
\zhlipsum[1]

$$\int^b_a f(x)\dif x = F(b) - F(a)$$

\SectionEndLine				   %每一小节结束时,使用此命令生成表格结束的横线


%%%%%%%%%%%%%%%%%%%%%%%%%%%%%%%%%%%%%%%%%%%%%%%%%%%%%%%%%
%% 第三部分: 研究现状与参考文献
%%%%%%%%%%%%%%%%%%%%%%%%%%%%%%%%%%%%%%%%%%%%%%%%%%%%%%%%%%%
\section{选题的研究现状及主要参考文献}
% 在这里撰写研究现状

\subsection{问题的起源}
\zhlipsum[1]

\subsection{新近进展}

\zhlipsum[1]

\subsection{存在的问题}

\zhlipsum[1]

\subsection{自己写合适的小节标题吧}

\zhlipsum[1]

\begin{itemize}
	\item[\ding{172}] 期刊论文, 例如\cite{XmDeng2007,Brown1984,ZHY2017RS,ZHY2012manifold}以及\cite{Wing2006CT}
	\item[\ding{173}] 会议论文, 例如\cite{Geyer1999,Branislav2004}
	\item[\ding{174}] 专利, 例如\cite{LongMA2015P}
	\item[\ding{175}] 书籍, 例如\cite{CDIO2014,ZHY2022}
	\item[\ding{176}] 在线资源, 例如\cite{SFMedu,OpenMVS,HainnuThesis}
	\item[\ding{177}] 技术报告,例如\cite{Sussman2005}
	\item[\ding{178}] 学位论文,例如\cite{ZHY2002,Xue2006,lzj-zhy-MS-thesis-2018,zls-zhy-MS-thesis-2016,zhy-2011-disertation}

\end{itemize}
\ckwx
\bibliography{reference.bib}
\SectionEndLine				   %每一小节结束时,使用此命令生成表格结束的横线
%%%%%%%%%%%%%%%%%%%%%%%%%%%%%%%%%%%%%%%%%%%%%%%%%%%%%%%%%
%% 第四部分: 研究内容/创新点/重难点与研究思路
%%%%%%%%%%%%%%%%%%%%%%%%%%%%%%%%%%%%%%%%%%%%%%%%%%%%%%%%%%%

\section{拟研究的主要内容、创新点、重难点及研究思路}
% 在这里撰写创新点

\subsection{主要内容}

\zhlipsum[2]

\subsection{创新点(独创性,原创性)}

\zhlipsum[1]

\subsection{研究/开发/设计/...的重点/难点}

\zhlipsum[1]

\subsection{解决问题的思路/技术路线/构思}

\zhlipsum[1]

\SectionEndLine				   %每一小节结束时,使用此命令生成表格结束的横线
%%%%%%%%%%%%%%%%%%%%%%%%%%%%%%%%%%%%%%%%%%%%%%%%%%%%%%%%%
%% 第五部分: 研究进程安排
%%%%%%%%%%%%%%%%%%%%%%%%%%%%%%%%%%%%%%%%%%%%%%%%%%%%%%%%%%%
\section{研究进程安排}
% 在这里撰写研究进程安排

这个部分适合列条目或用表格说事,这样比较简明扼要.



\SectionEndLine				   %每一小节结束时,使用此命令生成表格结束的横线
%%%%%%%%%%%%%%%%%%%%%%%%%%%%%%%%%%%%%%%%%%%%%%%%%%%%%%%%%
%% 第六部分: 其它说明
%%%%%%%%%%%%%%%%%%%%%%%%%%%%%%%%%%%%%%%%%%%%%%%%%%%%%%%%%%%
\section{其他说明}
% 在这里撰写其他说明

这部分因人而已,可以有,也可以无.


\SectionEndLine				   %每一小节结束时,使用此命令生成表格结束的横线
\section{指导教师意见}
% 在这里撰写指导教师意见

这部分留给指导教师写审核意见并签字. 
\begin{flushright}
	指导教师签名:\makebox[6em][l]{\leaderSign}\\
	\leaderSignDate
\end{flushright}



\SectionEndLine				   %每一小节结束时,使用此命令生成表格结束的横线
\section{专业毕业论文指导小组意见}
% 在这里撰写专业毕业论文指导小组意见
\vspace*{5cm}


\begin{flushright}
	毕业论文与毕业设计指导小组组长签字:\makebox[6em][l]{\leaderSign}\\
	\leaderSignDate
\end{flushright}


\end{document}
